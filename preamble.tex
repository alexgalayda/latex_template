\usepackage[T2A]{fontenc}			% кодировка
\usepackage[utf8]{inputenc}			% кодировка исходного текста
\usepackage[english,russian]{babel}		% локализация и переносы

\usepackage{extsizes}				% Возможность сделать 14-й шрифт
\usepackage{indentfirst}			% Сделать отступ для первого абзаца
\usepackage{amsmath,amsfonts,amssymb,amsthm,mathtools} % AMS

\usepackage{epigraph}				%эпиграф
\setlength\epigraphwidth{.6\textwidth}

% Секции без номеров (введение, заключение...), вместо section*{}
\newcommand{\anonsection}[1]{
	\phantomsection % Корректный переход по ссылкам в содержании
	\section*{#1}
	\addcontentsline{toc}{section}{#1}
}

%%%%%%%%%% Иначе не работают вообще никакие переходы
\usepackage{hyperref}
\usepackage[usenames,dvipsnames,svgnames,table,rgb]{xcolor}
\definecolor{linkcolor}{HTML}{799B03} % цвет ссылок
\definecolor{urlcolor}{HTML}{799B03} % цвет гиперссылок

\hypersetup{%
	pdfstartview=FitH,%
	linkcolor=blue,%
	citecolor=black,
	colorlinks=true}

\usepackage{csquotes} % Инструменты для ссылок

\usepackage[backend=biber,bibencoding=utf8,sorting=ynt,maxcitenames=2,style=authoryear,autocite=inline]{biblatex} %Для библиографии

\usepackage[most]{tcolorbox} % для управления цветом
\definecolor{block-gray}{gray}{0.90} % уровень прозрачности (1 - максимум)
\newtcolorbox{grayquote}{colback=block-gray,grow to right by=-10mm,grow to left by=-10mm, boxrule=0pt,boxsep=0pt,breakable} % настройки области с изменённым фоном

\usepackage[ruled,vlined]{algorithm2e}

\usepackage{float} %для того, чтобы фиксировать изображение
\usepackage{url}

\usepackage{pgf,tikz}
\usetikzlibrary{arrows}

\usepackage[acronym]{glossaries}
